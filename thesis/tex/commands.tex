% در این فایل، دستورها و تنظیمات مورد نیاز، آورده شده است.
%-------------------------------------------------------------------------------------------------------------------

\usepackage{stmaryrd}
\usepackage{amssymb,amsmath,amsthm}
\usepackage{mathrsfs}
\usepackage [justification=centering]{caption}
%ایجاد فرورفتگی در اولین پاراگراف هر بخش
\usepackage{indentfirst}
% بسته‌ای برای تنطیم حاشیه‌های بالا، پایین، چپ و راست صفحه
\usepackage[top=30mm, bottom=30mm, left=25mm, right=35mm]{geometry}
% بسته‌‌ای برای ظاهر شدن شکل‌ها و تصاویر متن
\usepackage{graphicx}
%\usepackage{mdframed}
\usepackage{caption}
%\usepackage{subcaption}
\usepackage{rotating}
\usepackage{pdflscape}
%\usepackage{subcaption}
\usepackage{mwe}
\usepackage{subfig}
%\usepackage{subcaption}
\usepackage{graphicx}
%\usepackage{tabu}
\usepackage{multirow}
\usepackage{siunitx}
\usepackage{booktabs}
\usepackage{hhline}
\usepackage{setspace}
%\usepackage{rotating}
%\usepackage{underscore}
\usepackage[usenames,dvipsnames]{color}
\usepackage{bbding}
\usepackage{xecolor}
\usepackage{bbold}

 
\usepackage{rotating}
\setlength{\rotFPtop}{0pt plus 1fil}


%\usepackage{xcolor}
\def\BibTeX{{\rm B\kern-.05em{\sc i\kern-.025em b}\kern-.08em
    T\kern-.1667em\lower.7ex\hbox{E}\kern-.125emX}}

% بسته‌های نوشتن شبه کد و الگوریتم
%\usepackage{algorithm} 
%\usepackage{algorithmic} 
\usepackage[algochapter,linesnumbered,ruled,vlined]{algorithm2e}
% بسته‌ای برای رسم کادر
\usepackage{framed} 
% بسته‌‌ای برای چاپ شدن خودکار تعداد صفحات در صفحه «معرفی پایان‌نامه»
\usepackage{lastpage}
% بسته‌‌ای برای ایجاد دیاگرام‌های مختلف
\usepackage[all]{xy}
\usepackage{tikz}

%\usepackage{zref-perpage}
%\zmakeperpage{footnote}

\usepackage{zref-abspage}
%\zmakeperpage{footnote}

\usepackage{perpage}
%\MakePerPage{footnote}

\usetikzlibrary{topaths}
\tikzset{terminal/.style={
							% The shape:
							rectangle ,minimum size =6mm,rounded corners=4mm,
							% The rest
							%very thick,draw=black!50,
							top color=green!10,bottom color=green!40,
							font=\ttfamily}}
	\renewcommand{\baselinestretch}{2}\normalsize


% بسته‌ و دستوراتی برای ایجاد لینک‌های رنگی با امکان جهش
%\usepackage{fancyref}
\usepackage[linktocpage=true,colorlinks,pagebackref=true,linkcolor=blue,citecolor=magenta]{hyperref}
% چنانچه قصد پرینت گرفتن نوشته خود را دارید، خط بالا را غیرفعال و  از دستور زیر استفاده کنید چون در صورت استفاده از دستور زیر‌‌، 
% لینک‌ها به رنگ سیاه ظاهر خواهند شد که برای پرینت گرفتن، مناسب‌تر است
%\usepackage[pagebackref=false]{hyperref}

%بسته‌ی لازم برای مراجع
\usepackage[sort&compress,numbers,square]{natbib}

% بسته‌ لازم برای تنظیم سربرگ‌ها
\usepackage{fancyhdr}

%\usepackage{labels}
\usepackage{nameref}
% بسته‌ای برای ظاهر شدن «مراجع» و «نمایه» در فهرست مطالب
\usepackage[nottoc]{tocbibind}
% دستورات مربوط به ایجاد نمایه
\usepackage{makeidx}
\makeindex
%%%%%%%%%%%%%%%%%%%%%%%%%%
% فراخوانی بسته زی‌پرشین و تعریف قلم فارسی و انگلیسی
%\usepackage{xepersian}
\usepackage[localise=on]{xepersian}
\settextfont[Scale=1.15]{XB Niloofar.ttf}
%\setiranicfont[Scale=1]{PersianModern-Oblique}
\setiranicfont[Scale=1.15]{XB Niloofar.ttf}
% چنانچه می‌خواهید اعداد در فرمول‌ها، انگلیسی باشد، خط زیر را غیرفعال کنید
%\setdigitfont{Yas}
%\setdigitfont[Scale=.85]{Persian Modern}
%\setdigitfont[Scale=1.0]{XB Zar}
%\KashidaOn
%تنظیم فاصله‌ی خطوط و پاراگراف‌ها
%\linespread{1.3}
\setlength{\parskip}{1ex plus 0.5ex minus 0.2ex}
%%%%%%%%%%%%%%%%%%%%%%%%%%
% تعریف قلم‌های فارسی و انگلیسی اضافی برای استفاده در بعضی از قسمت‌های متن
%\defpersianfont\nastaliq[Scale=2.02]{IranNastaliq}
\defpersianfont\nastaliq[Scale=2.02]{XB Niloofar.ttf}
\defpersianfont\chapternumber[Scale=1.02]{XB Niloofar.ttf}
\defpersianfont\titr[Scale=2.02]{XB Titre.ttf}
%\defpersianfont\dav[Scale=2]{B Davat}
\defpersianfont\dav[Scale=2.02]{XB Niloofar.ttf}
\graphicspath{ {./images/} }
%%%%%%%%%%%%%%%%%%%%%%%%%%
% دستوری برای حذف کلمه «چکیده»
%%\renewcommand{\abstractname}{}
% دستوری برای حذف کلمه «abstract»
\renewcommand{\latinabstract}{}
% دستوری برای تغییر نام کلمه «اثبات» به «برهان»
\renewcommand\proofname{\textbf{برهان}}
% دستوری برای تغییر نام کلمه «ااب‌نامه» به «مراجع»
\renewcommand{\bibname}{مراجع}
%استفاده از کلمه‌ی نمودار به جای شکل
%\renewcommand{\figurename}{نمودار}
% تنظیمات مربوط به الگوریتم‌ها
\renewcommand{\algorithmcfname}{الگوریتم}      % used for title
\renewcommand{\algorithmautorefname}{الگوریتم} % used for autoref
\renewcommand{\listalgorithmcfname}{لیست الگوریتم‌ها} % used for list of algorithms

\newcommand\mycommfont[1]{\footnotesize\ttfamily\textcolor{blue}{#1}}
\SetCommentSty{mycommfont}

% دستوری برای تعریف واژه‌نامه انگلیسی به فارسی
\newcommand\persiangloss[2]{#1\dotfill\lr{#2}\\}
% دستوری برای تعریف واژه‌نامه فارسی به انگلیسی 
\newcommand\englishgloss[2]{#2\dotfill\lr{#1}\\}
%%%%%%%%%%%%%%%%%%%%%%%%%%
% تعریف و نحوه ظاهر شدن عنوان قضیه‌ها، تعریف‌ها، مثال‌ها و ...
\theoremstyle{definition}
\newtheorem{definition}{تعریف}[section]
%\newtheorem{deduct}[definition]{}
\newtheorem{theorem}[definition]{قضیه}
\newtheorem{lemma}[definition]{لم}
\newtheorem{proposition}[definition]{گزاره}
\newtheorem{corollary}[definition]{نتیجه}
\newtheorem{remark}[definition]{ملاحظه}
\newtheorem{convention}[definition]{قرارداد}
\newtheorem{example}[definition]{مثال}
%%%%%%%%%%%%%%%%%%%%%%%%%%
% تعریف دستورات جدید برای خلاصه نویسی و راحتی کار در هنگام تایپ فرمول‌های ریاضی
\newcommand{\mA}{\mathcal{A}}% مجموعه‌ی عامل‌ها
\newcommand{\mB}{\mathcal{B}}% زیرمجموعه‌ای از عامل‌ها
\newcommand{\mL}{\mathcal{L}}% زبان
\newcommand{\mF}{\mathcal{F}}% سیگما جبر
\newcommand{\mP}{\mathcal{P}}% تخصیص احتمالاتی
\newcommand{\mK}{\mathcal{K}}% منطق K
\newcommand{\mS}{\mathcal{S}}% منطق S5
\newcommand{\mbbP}{\mathbb{P}}% مجموعه‌ی گزاره‌های اتمی
\newcommand{\mbbQ}{\mathbb{Q}}% مجموعه‌ی اعداد گویا
\newcommand{\mbbM}{\mathbb{M}}% مجموعه‌ی مدل‌های شناختی احتمالاتی
\newcommand{\mbbK}{\mathbb{K}}% کلاس مدل‌های کریپکی
\newcommand{\mbbS}{\mathbb{S}}% کلاس مدل‌های شناختی
\newcommand{\mbP}{\mathbf{P}}% نماد تابعی احتمالاتی
\newcommand{\xra}{\xrightarrow}
\newcommand{\scr}{\scriptscriptstyle}
\newcommand{\bp}{\begin{proof}}
\newcommand{\ep}{\end{proof}}
%\newcommand{\close}{\begin{latin}\noindent $\square$ \end{latin}}
%%%%%%%%%%%%%%%%%%%%%%%%%%%%%%%%%%%%%%%%%%%
%تعریف اعداد لاتین برای زمانی که از اعداد فارسی در متن استفاده می‌کنیم
\def\0{\textrm{\lr{0}}}
\def\1{\textrm{\lr{1}}}
\def\2{\textrm{\lr{2}}}
\def\3{\textrm{\lr{3}}}
\def\4{\textrm{\lr{4}}}
\def\5{\textrm{\lr{5}}}
\def\5{\textrm{\lr{6}}}
\def\5{\textrm{\lr{7}}}
\def\5{\textrm{\lr{8}}}
\def\5{\textrm{\lr{9}}}
%%%%%%%%%%%%%%%%%%%%%%%%%%%%%%%%%%%%%%%%%%%
\def\reduction{تحویل}
\def\observation{مشاهده‌محور}
\def\prior{پیشینی}
%%%%%%%%%%%%%%%%%%%%%%%%%%%%%%%%%%%%%%%%%%%
% براساس نسخه‌های از ۱.۲.۵ به بعد بسته‌ی bidi در توابع زیر باید توجه داشت:
% که l یعنی شروع نوشتار از ابتدای خط یا جعبه،
% و r یعنی شروع نوشتار از انتهای خط یا جعبه، 
% و s یعنی شروع نوشتار از وسط خط یا جعبه.
\newdimen\xleftright
\xleftright=\textwidth
\advance \xleftright by -10.5cm
\newcommand{\leftright}[3]{%
\noindent
\makebox[4 cm][r]{#1}
\makebox[\xleftright][s]{}
\makebox[5.5 cm][l]{#2}
\makebox[1 cm][l]{#3}%
}
\newcommand{\leftrightb}[2]{%
\noindent
\makebox[4 cm][r]{#1}
\makebox[\xleftright][s]{}
\makebox[6.5 cm][l]{#2}%
}
\newcommand{\semanticsa}[2]{%
\noindent
\makebox[4.3 cm][r]{#1}
\makebox[\xleftright][c]{اگر و فقط اگر}
\makebox[6.2 cm][l]{#2}%
}
\newcommand{\semanticsb}[2]{%
\noindent
\makebox[3.5 cm][l]{#1}
\makebox[\xleftright][c]{اگر و فقط اگر}
\makebox[7 cm][r]{#2}%
}
%%%%%%%%%%%%%%%%%%%%%%%%%%%%
% دستورهایی برای سفارشی کردن سربرگ صفحات
\csname@twosidetrue\endcsname
\pagestyle{fancy}
\fancyhf{} 
\fancyhead[RE,LO]{\thepage}
\fancyhead[LE]{\small\iranicfamily\leftmark}
\fancyhead[RO]{\small\iranicfamily\rightmark}
\renewcommand{\chaptermark}[1]{%
\markboth{\thechapter.\ #1}{}}
%%%%%%%%%%%%%%%%%%%%%%%%%%%%%
% دستورهایی برای سفارشی کردن صفحات اول فصل‌ها
\makeatletter
\newcommand\mycustomraggedright{%
 \if@RTL\raggedleft%
 \else\raggedright%
 \fi}
\def\@part[#1]#2{%
\ifnum \c@secnumdepth >-2\relax
\refstepcounter{part}%
\addcontentsline{toc}{part}{\thepart\hspace{1em}#1}%
\else
\addcontentsline{toc}{part}{#1}%
\fi
\markboth{}{}%
{\centering
\interlinepenalty \@M
\ifnum \c@secnumdepth >-2\relax
 \huge\bfseries \partname\nobreakspace\thepart
\par
\vskip 20\p@
\fi
\Huge\bfseries #2\par}%
\@endpart}
\def\@makechapterhead#1{%
\vspace*{-30\p@}%
{\parindent \z@ \mycustomraggedright %\@mycustomfont
\ifnum \c@secnumdepth >\m@ne
\if@mainmatter

\huge\bfseries \@chapapp\space {\chapternumber\thechapter}
\par\nobreak
\vskip 20\p@
\fi
\fi
\interlinepenalty\@M 
\Huge \bfseries #1\par\nobreak
\vskip 120\p@
}}
\makeatother
