%% در این فایل، عنوان پایان‌نامه، مشخصات خود، متن تقدیمی‌، ستایش، سپاس‌گزاری و چکیده پایان‌نامه را به فارسی، وارد کنید.
%% توجه داشته باشید که جدول حاوی مشخصات پایان‌نامه/رساله و همچنین، مشخصات داخل آن، به طور خودکار، درج می‌شود.
%%%%%%%%%%%%%%%%%%%%%%%%%%%%%%%%%%%%%

\university{شهید بهشتی}
\faculty{علوم ریاضی}
\degree{کارشناسی ارشد} 
\subject{علوم داده‌ها و کامپیوتر}
\field{داده‌کاوی}
\title{یادگیری خودنظارتی در شبکه های عصبی اسپایکی(ضربه‌ای) }
\firstsupervisor{دکتر هادی فراهانی}
\secondsupervisor{دکتر سعیدرضا خردپیشه}
\name{یگانه}
\surname{بهاری اصل}
\thesisdate{شهریور ماه ۱۴۰۲}

\newpage
\thispagestyle{empty}
\keywords{شبکه های عصبی اسپایکی(ضربه‌ای)  , یادگیری خودنظارتی , دسته بندی داده }
\fa-abstract{ 
در این پایان‌نامه، به بررسی و معرفی شبکه‌های عصبی اسپایکی (ضربه‌ای) به‌عنوان نسل جدیدی از شبکه‌های عصبی می‌پردازیم. این شبکه‌ها، الهام گرفته از ساختار مغز، اطلاعات را از طریق قطارهای اسپایکی که به‌طور دقیق زمان‌بندی شده‌اند، کدگذاری و پردازش می‌کنند. برخلاف باقی شبکه‌های عصبی، در این نوع از شبکه نورون‌ها به جز مواقع تولید اسپایک در حال استراحت هستند و این خاصیت باعث بهینه بودن ساختار این شبکه‌ها و کم‌ترین استفاده از حافظه می‌شود.
در این پژوهش، به بررسی روش‌های یادگیری خودنظارتی در شبکه‌های عصبی اسپایکی می‌پردازیم. این روش قابلیت دارد که با استفاده از تکنیک‌هایی مانند افزایش داده ، یادگیری را در شبکه بدون نیاز به داده برچسب دار انجام دهد.
در نهایت، هدف از این پژوهش ساخت یک شبکه عصبی اسپایکی بهینه با قابلیت استفاده از روش‌های یادگیری خودنظارتی است تا بدون نیاز به برچسب داده، نمایش‌هایی از مجموعه داده‌های MNIST و CIFAR10 تولید کرده و از این نمایش‌ها برای وظایف مختلفی مانند دسته‌بندی نظارت شده استفاده شود. این پژوهش تلاش می‌کند تا با ساخت این نمایش‌ها، در نهایت دقت بیشتری را در عملکرد شبکه ساخته شده بر روی روش‌های دسته‌بندی روی مجموعه داده‌ها ایجاد کند.}



\vtitle
\newpage
\thispagestyle{empty}
\mbox{}
\newpage
\thispagestyle{empty}
\ \\ \\ \\ \\ \\ \\ \\
{\dav
\begin{center}
كلية حقوق اعم از چاپ و تكثير، نسخه برداري ، ترجمه، اقتباس و ... از اين پايان نامه براي دانشگاه شهيد بهشتي محفوظ است.
 نقل  مطالب با ذكر مأخذ آزاد است.
\end{center}
}
\newpage
\thispagestyle{empty}
\mbox{}
\newpage
\thispagestyle{empty}
{\nastaliq \footnotesize
با ارادت و ادب، از تلاش‌های بی‌دریغ اساتید محترم، جناب آقای دکتر خردپیشه و جناب آقای دکتر فراهانی که در انجام این پایان‌نامه با بی‌دریغی راهنمایی‌های بسیاری ارائه داده‌اند، کمال قدردانی را دارم. 

همچنین، از همسر عزیز,  پدر گرامی و مادربزرگ مهربانم که با محبت و پشتیبانی‌های بی‌اندازه، دلسوزانه در کنارم بوده‌اند، تشکر عمیق را ابراز می‌کنم.
}
\signature 
\newpage
\thispagestyle{empty}
{\nastaliq \footnotesize
تقدیم به مادرم...
}

~~~
\newpage
{\small
	\abstractview
}
\newpage
\thispagestyle{empty}
\clearpage
~~~









%% دانشگاه خود را وارد کنید
%\university{شهید بهشتی}
%% دانشکده، آموزشکده و یا پژوهشکده  خود را وارد کنید
%\faculty{علوم ریاضی}
%% گروه آموزشی خود را وارد کنید
%\degree{کارشناسی ارشد} 
%% گروه آموزشی خود را وارد کنید
%\subject{علوم داده‌ها و کامپیوتر}
%% گرایش خود را وارد کنید
%\field{داده‌کاوی}
%% عنوان پایان‌نامه را وارد کنید
%\title{یادگیری خودنظارتی در شبکه های عصبی ضربه ‌ای}
%% نام استاد(ان) راهنما را وارد کنید
%\firstsupervisor{استاد راهنمای اول}
%\secondsupervisor{استاد راهنمای دوم}
%% نام استاد(دان) مشاور را وارد کنید. چنانچه استاد مشاور ندارید، دستور پایین را غیرفعال کنید.
%% نام پژوهشگر را وارد کنید
%\name{نام}
%% نام خانوادگی پژوهشگر را وارد کنید
%\surname{نام خانوادگی}
%% تاریخ پایان‌نامه را وارد کنید
%\thesisdate{تیرماه ۱۴۰۲}
%
%
%
%\newpage
%\thispagestyle{empty}
%% کلمات کلیدی پایان‌نامه را وارد کنید
%\keywords{کلمات کلیدی}
%% چکیده پایان‌نامه را وارد کنید
%\fa-abstract{ 
%	چکیده
%}
%
%\newpage
%\thispagestyle{empty}
%\mbox{}
%\newpage
%\thispagestyle{empty}
%\ \\ \\ \\ \\ \\ \\ \\
{\dav
\begin{center}
كلية حقوق اعم از چاپ و تكثير، نسخه برداري ، ترجمه، اقتباس و ... از اين پايان نامه براي دانشگاه شهيد بهشتي محفوظ است.
 نقل  مطالب با ذكر مأخذ آزاد است.
\end{center}
}
%\newpage
%\thispagestyle{empty}
%\mbox{}
%\newpage
%\thispagestyle{empty}
%% سپاس‌گزاری
%{\nastaliq \footnotesize
%	خرسندم که انسان‌های بزرگی را در مسیر زندگی تجربه نمودم و از آنان دانش و بینش آموختم. 
%	%از آنان سپاس‌گزارم و برایشان زیبایی، تندرستی و آرامش آر
%	
%	
%	
%	
%
%	
%	
%	
%
%%\newpage
%%\thispagestyle{empty}
%%% کلمات کلیدی پایان‌نامه را وارد کنید
%%\keywords{ lkfdemlkrglکلمات کلیدی}
%%% چکیده پایان‌نامه را وارد کنید
%%\fa-abstract{ 
%%	
%%	چکیده
%%}
%%
%%\newpage
%%\thispagestyle{empty}
%%\vtitle
%%\newpage
%%\thispagestyle{empty}
%%\clearpage
%%~~~
%%\newpage
%%\thispagestyle{empty}
%%\ \\ \\ \\ \\ \\ \\ \\
{\dav
\begin{center}
كلية حقوق اعم از چاپ و تكثير، نسخه برداري ، ترجمه، اقتباس و ... از اين پايان نامه براي دانشگاه شهيد بهشتي محفوظ است.
 نقل  مطالب با ذكر مأخذ آزاد است.
\end{center}
}
%%\newpage
%%\thispagestyle{empty}
%%\clearpage
%%~~~
%%\newpage
%%\thispagestyle{empty}
%%% سپاس‌گزاری
%%{ \nastaliq \footnotesize
%%خرسندم که انسان‌های بزرگی را در مسیر زندگی تجربه نمودم و از آنان دانش و بینش آموختم. 
%%%از آنان سپاس‌گزارم و برایشان زیبایی، تندرستی و آرامش آرزومندم. 
%%این صفحه مجالی‌ست که نیکی‌ها و بزرگوارهای برخی از این عزیزان را با افتخار پاس بدارم.
%%}
%%\signature 
%%\newpage
%%\thispagestyle{empty}
%%\clearpage
%%~~~
%%\newpage
%%{\small
%%\abstractview
%%}
%%\newpage
%%\thispagestyle{empty}
%%\clearpage
%%~~~
%%%\newpage
