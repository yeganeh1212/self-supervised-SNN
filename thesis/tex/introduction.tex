\clearpage
\phantomsection
\addcontentsline{toc}{chapter}{مقدمه}
\chapter*{مقدمه}\markboth{مقدمه}{مقدمه}


شبکه‌های عصبی اسپایکی به دلیل توانایی پردازش اطلاعات زمانی، مصرف کم انرژی و قابلیت بیولوژیکی بالا، در حال رشد هستند. این شبکه‌ها با الهام از مکانیزم نورون‌های بیولوژیکی، به جای مقادیر حقیقی، با استفاده از پالس‌های الکتریکی و اسپایک‌ها، اطلاعات را منتقل می‌کنند.
\citep{taherkhani2020review}
 یکی از اصلی‌ترین دلایل استفاده از شبکه‌های عصبی اسپایکی، شباهت این شبکه‌ها به نورون‌های بیولوژیکی انسان و حیوانات است. این شبکه‌ها به معماری و عملکرد نورون‌های واقعی پایبند بوده و توانایی در نمایش پردازش اطلاعات به صورت الکتریکی و با تولید اسپایک‌ها (پالس‌های الکتریکی کوتاه) دارند. این شباهت به نورون‌های بیولوژیکی به ما امکان می‌دهد تا رفتارها و ویژگی‌هایی که در مغز و سیستم عصبی واقعی مشاهده می‌شود را به دقت بیشتری شبیه‌سازی کنیم.
\citep{rafi2021brief}
 یکی از مزایای مهم شبکه‌های عصبی اسپایکی این است که به دلیل فرآیند انتقال اطلاعات با اسپایک‌های الکتریکی، مصرف کمتری از انرژی نیاز دارند. این ویژگی بسیار مهم است، به ویژه در کاربردهایی مانند رباتیک که نیاز به سیستم‌های با مصرف انرژی پایین دارند .
  همچنین این شبکه‌ها از نظر ساختاری و عملکردی به بیولوژی نزدیک‌تر هستند و این قابلیت برای تحقیقات در زمینه علوم اعصاب بسیار ارزشمند است.
\citep{bing2018survey}

از طرفی یادگیری خودنظارتی یک حوزه پژوهشی در حال رشد است که در سال‌های اخیر با نتایج چشمگیر، به ویژه در زمینه پردازش تصویر، در حال پیشروی است. این روش یادگیری بر روی داده‌های بدون برچسب اعمال می‌شود و از طریق یک فرآیند افزایش داده بر روی داده‌های ورودی، سعی در ایجاد نمایشی از هر داده دارد که بیشترین تمایز از سایر داده‌ها را داشته باشد. به این صورت که در هر مرحله، شبکه آموزش می‌بیند که نمایش‌های یکسان از حالت‌های یک داده را ایجاد کند، در حالی که نمایش ایجاد شده برای هر داده، از دیگر داده‌ها متمایز باشد.
\citep{gidaris2018unsupervised}

با توجه به ماهیت نوآورانه روش‌های یادگیری خودنظارتی و هم‌چنین شبکه‌های عصبی اسپایکی به عنوان نسل جدیدی از شبکه‌های عصبی, تاکنون پژوهش‌های زیادی در حوزه پیاده سازی این روش بر روی شبکه‌های عصبی اسپایکی صورت نگرفته و در هر یک از این دو حوزه به صورت مستقل مطالعاتی انجام شده است. 
در روش‌های یادگیری خودنظارتی با توجه به اینکه این روش نیاز به داده برچسب‌دار را تا حد خوبی کم می‌کند در حوزه‌های متنوعی به خصوص در داده‌های پزشکی که فرآیند برچسب زدن داده بسیار پرهزینه است, مطالعات جالب توجهی انجام شده است.
\citep{jaiswal2020survey}
\citep{shurrab2022self}
 از طرفی از شبکه‌های عصبی اسپایکی نیز با توجه به ماهیت بیولوژیکی آن‌ها و بهینه بودن در مصرف انرژی, در حوزه‌هایی مانند رباتیک و مهندسی نورومورفیک و هم‌چنین تشخیص الگو در داده‌های مغزی پژوهش‌هایی صورت گرفته است.
\citep{huynh2022implementing}
\citep{rathi2023exploring}

در این پروژه، قصد داریم با مدل‌سازی یک شبکه عصبی اسپایکی و استفاده از داده ورودی به عنوان یک جریان ثابت به شبکه، و شبیه‌سازی نورون‌ها با یک مدار الکتریکی، و پیاده‌سازی مدل نورونی یک‌پارچه و نشتی آتش\LTRfootnote{\lr{Leaky integerate and fire}}، و با استفاده از روش گرادیان مجازی\LTRfootnote{\lr{Surrogate gradient}} در یادگیری و پیاده‌سازی روش یادگیری خودنظارتی بر روی این شبکه یک ابزار قدرتمند و بهینه برای تولید نمایش‌های اسپایکی از داده‌های بدون برچسب بسازیم. به این صورت که اطلاعات اصلی هر نمونه از داده در این نمایش‌ها ذخیره شود و به گونه‌ای ساخته شود که هر نمایش بیشترین تمایز را از نمایش دیگر داده‌ها داشته باشد و در نهایت این نمایش‌‌های اسپایکی ساخته شده می‌توانند در انجام وظایف متنوعی مفید باشند.

این ابزار قادر است وظایفی مانند دسته بندی نظارت‌شده و تشخیص الگو در داده‌های تصویر بدون نیاز به برچسب انجام دهد و بسیار در کاهش هزینه و زمان مفید باشد. علاوه بر این با توجه به ماهیت شبکه‌های عصبی اسپایکی می‌توان از این ابزار ساخته شده برای وظایفی مانند ساخت ابزار‌های نورومورفیک و پردازش داده‌های مغزی, بدون نیاز به هزینه و زمان برای ساخت داده‌ی برچسب‌دار استفاده کرد .








